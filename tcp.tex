\documentclass[a4paper,11pt]{article}
\usepackage{a4wide}
\usepackage{fullpage}
\usepackage[utf8x]{inputenc}
%\usepackage[slovene]{babel}
%\selectlanguage{slovene}
\usepackage[toc,page]{appendix}
\usepackage[pdftex]{graphicx} % za slike
\usepackage{setspace}
\usepackage{color}
\definecolor{light-gray}{gray}{0.95}
\usepackage{listings} 

\begin{document}
\chapter{Modeliranje TCP/IP omrežij}
%\chapterauthor{Matevž Ogrinc, Robert Modic, Andreja Kovačič}

\section{Gradniki za realizacijo IPv4 omrežij}
	\begin{itemize}
		\item\textit{IPv4}: glavni modul, ki implementira IP protokol(RFC 791). Modul izvaja enkapsulacijo, dekapsulacijo, fragmentacijo in defragmentacijo in usmerjanje IP datagramov. Prispeli paketi se posredujejo v ARP modul. Privzeta predpostavka modula je, da se vsi paketi obravnavajo enako dolgo, brez prioritet. Za drugačne nastavitve je treba spremeniti implementacijo.
		\item\textit{IPv4RoutingTable}: pomožni modul, ki skrbi za usmerjevalne tabele vozlišč. Vsak gostitelj in usmerjevalki vsebuje eno kopijo. Modulu IPv4 dostavlja podatke o najboljših poteh, posodabljajo pa ga daemoni RIP, OSPF, Manet in drugih protokolov. Modul nima vrat, vsa komunikacija z njim poteka skozi klice funkcije (predvsem read in update). Ko je na voljo več poti za dan naslov, se upošteva pot:
		z najdaljšim ujemanjem naslova,z najbolj specifično multicast skupino, z najmanjšo metriko (ceno poti).
		\item\textit{ICMP}: modul generira ICMP(RFC 792) pakete, podpira echo aplikacije. ICMP protokol se uporablja za spetno javljnje napak in diagnostiko. Ker uporablja protokol IP, spada med transportne protokole, vendar se za razliko od TCP/UDP protokolov ne uporablja za prenos uporabnikovih podatkov.
		\item\textit{ARP}: izvaja dinamično prevajanje med lokalnimi naslovi(tipično IP) in strojnimi naslovi(MAC), implementira RFC 826. Inet-ova implementacija podpira samo preslikavo IP-MAC.
		\item\textit{IGMPv2}: Modul generira in procesira multicast sporočila o članstvu odjemalcev v multicast skupine. Podatke posreduje usmerjevalnikom v omrežju. Ko se vmesnik gostitelja želi včlaniti v skupino, pošlje IGMP poročilo vmesniku multicast usmerjevalnika, ta ga obdela in posodobi tabelo naslovnikov multicast sporočil. Podoben postopek je ob izstopu gostitelja iz skupine. 
	\end{itemize}

Moduli so sestavljeni v \textit{IPv4NetworkLayer}, ki predstavlja celotno omrežno plast. Ima vrata za TCP, UDP, SCTP, RSVP in druge protokole.  Nanj se lahko povežejo omrežni vmesniki: Ethernet, PPP, Wlan ali drugi zunanji vmesniki. Modul se uporablja za gradnjo gostiteljev(hosts) in usmerjevalnikov (routers).
\end{document}